\documentclass[]{article}
\usepackage{amsfonts}
\usepackage{amsmath}
\usepackage{listings}
\usepackage{color}

\definecolor{mygreen}{rgb}{0,0.6,0}
\definecolor{mygray}{rgb}{0.5,0.5,0.5}
\definecolor{mymauve}{rgb}{0.58,0,0.82}
\definecolor{mypurple}{rgb}{0.6, 0.1, 0.1}

\lstset{ %
	backgroundcolor=\color{white},   % choose the background color; you must add \usepackage{color} or \usepackage{xcolor}; should come as last argument
	basicstyle=\footnotesize,        % the size of the fonts that are used for the code
	breakatwhitespace=false,         % sets if automatic breaks should only happen at whitespace
	breaklines=true,                 % sets automatic line breaking
	captionpos=b,                    % sets the caption-position to bottom
	commentstyle=\color{mygreen},    % comment style
	deletekeywords={...},            % if you want to delete keywords from the given language
	escapeinside={\%*}{*)},          % if you want to add LaTeX within your code
	extendedchars=true,              % lets you use non-ASCII characters; for 8-bits encodings only, does not work with UTF-8
	frame=single,	                   % adds a frame around the code
	keepspaces=true,                 % keeps spaces in text, useful for keeping indentation of code (possibly needs columns=flexible)
	keywordstyle=\color{blue},       % keyword style
	language=Python,                 % the language of the code
	morekeywords={*,...},           % if you want to add more keywords to the set
	numbers=left,                    % where to put the line-numbers; possible values are (none, left, right)
	numbersep=5pt,                   % how far the line-numbers are from the code
	numberstyle=\tiny\color{mygray}, % the style that is used for the line-numbers
	rulecolor=\color{black},         % if not set, the frame-color may be changed on line-breaks within not-black text (e.g. comments (green here))
	showspaces=false,                % show spaces everywhere adding particular underscores; it overrides 'showstringspaces'
	showstringspaces=false,          % underline spaces within strings only
	showtabs=false,                  % show tabs within strings adding particular underscores
	stepnumber=2,                    % the step between two line-numbers. If it's 1, each line will be numbered
	stringstyle=\color{mymauve},     % string literal style
	tabsize=2,	                   % sets default tabsize to 2 spaces
	title=\lstname,               % show the filename of files included with \lstinputlisting; also try caption instead of title
	emph={isPermutation},         
	emphstyle=\ttfamily\color{mypurple}
}

%opening
\title{Permutation Detection Algorithm with Linear Complexity}
\author{Ezra A. Singh}

\begin{document}

\maketitle

\begin{abstract}
The algorithm will use a product of indexed primes to generate permutation invariant value. This method allows for $O(N)$ possibly one of the fastest methods to date for this type of problem.
\end{abstract}

\section*{The Problem}
Given two sets $A$ and $B$ of symbols with finite length, how can we determine if set $A$ permutes set $B$. We will use the base 10 radix to define our set of symbols. Such that $A^{N}$ defines set of length $N$ where
\begin{equation}
A^{N}=\{a_{0}, a_{1}, a_{2}, ..., a_{N-2}, a_{N-1}\}
\end{equation}
\begin{equation}
a_{j} \in \{0,1, 2, 3, 4, 5, 6, 7, 8, 9\};\qquad j \in [0, N)
\end{equation}

Next we define a function $P$, the permutation function, which takes in two sets $A^{N}$ and $B^{N}$ and returns $1$ if the two sets permute each other else returns some other number. Our goal is to explicitly define an algorithm that resolves $P(A, B)$ in linear time.

\section*{The Permutation Algorithm}
Let us first map our symbol set to the set of prime numbers
\begin{equation}
B_{10} \longrightarrow \mathcal{P}^{10}
\end{equation}
\begin{equation}
\{0, 1, 2, 3, 4, 5, 6, 7, 8, 9\} \longrightarrow \{2, 3, 5, 7, 11, 13, 17, 19, 23, 29\}
\end{equation}
We will then introduce an indexing notation such that $\mathcal{P}_{0}=2$ and $\mathcal{P}_{9}=29$. Now given two sets $A$ and $B$ \footnote{We will omit the superscript because the length of $A$ and $B$ are implied to be the same, otherwise the sets trivially do not permute each other.} we can apply this mapping to each symbol within our sets creating what I call a projection into prime number space.
\begin{equation}
A = \{a_{1}, a_{2}, ... a_{N}\}  \longrightarrow \{\mathcal{P}_{a_{1}}, \mathcal{P}_{a_{2}}, ... \mathcal{P}_{a_{N}} \} = \mathcal{P}(A)
\end{equation}
We can define our permutation function as'
\begin{equation}
P(A, B) = \prod_{j=1}^{N} \frac{\mathcal{P}(a_{j})}{\mathcal{P}(b_{j})} = \frac{\mathcal{P}(a_{1}) \times \mathcal{P}(a_{2}) \times \dots \times \mathcal{P}(a_{N})}{\mathcal{P}(b_{1}) \times \mathcal{P}(b_{2}) \times \dots \times \mathcal{P}(b_{N})}
\end{equation}
Lets analyze this algorithm with two examples, the first example will demonstrate two sets that permute each other and the second will demonstrate two sets that do not.
\subsection*{Example 1}
\begin{equation}
	A = \{3, 3, 4, 1\}
\end{equation}
\begin{equation}
	B = \{3, 1, 3, 4\}
\end{equation}
\begin{equation}
P(A, B) = \frac{\mathcal{P}(3) \times \mathcal{P}(3) \times \mathcal{P}(4) \times \mathcal{P}(1)}{\mathcal{P}(3) \times \mathcal{P}(1) \times \mathcal{P}(3) \times \mathcal{P}(4)} = \frac{7 \times 7 \times 11 \times 3}{7 \times 3 \times \times 7 \times 11} = \frac{7^{2} \times 11 \times 3}{7^{2} \times 11 \times 3} = 1
\end{equation}
\subsection*{Example 2}
\begin{equation}
A = \{0, 2, 0, 2\}
\end{equation}
\begin{equation}
B = \{2, 3, 0, 0\}
\end{equation}
\begin{equation}
P(A, B) = \frac{\mathcal{P}(0) \times \mathcal{P}(2) \times \mathcal{P}(0) \times \mathcal{P}(2)}{\mathcal{P}(2) \times \mathcal{P}(3) \times \mathcal{P}(0) \times \mathcal{P}(0)} = \frac{2 \times 3 \times 2 \times 3}{3 \times 7 \times \times 2 \times 2} = \frac{2^{2} \times 3^{2}}{2^{2} \times 3 \times 7} = \frac{3}{7}
\end{equation}
\pagebreak
\section*{Implementation}
\subsection*{Python}
Complexity\footnote{$K$ represents the bit width of a floating point number in python which is implementation specific. For further information refer to the architecture of your system and the IEEE 754 standard}: $\mathcal{O}(K*(2^{K})*N) \;  \alpha \; \mathcal{O}(N);\quad K=[16, 24]$
\begin{lstlisting}
def isPermutation(A, B):
	primes = [2.0, 3.0, 5.0, 7.0, 11.0, 13.0, 17.0, 19.0, 23.0, 29.0]
	if not (len(A) - len(B)):
		product = 1.0
		N = len(A)
		for j in range(N):
			product *= primes[int(A[j])]
			product /= primes[int(B[j])]
		return product
	else:
		# trivial case
		return 0.0
\end{lstlisting}
With this implementation it is possible to detect permutations from an arbitrary amount of sets through daisy chaining
\begin{lstlisting}[frame=single]
# P(A, B, C)
isPermutation(A,B)*isPermutation(B,C)

# P(A, B, ...Y, Z)
isPermutation(A, B)*...*isPermutation(Y, Z)
\end{lstlisting}



\end{document}

